\documentclass[t,aspectratio=1610]{beamer}
\usepackage[orientation=portrait,size=a0,scale=1.8]{beamerposter}

\usepackage{Helmholtz-AI-poster}
\usepackage{tabularx}

% References
\usepackage[
    backend=biber,
    url=true
    ]{biblatex}
\addbibresource{references.bib}

% Use standard TeX math font.
%
% Copy or link /path/to/beamerfontthemeserif.sty into theme/, e.g.
%   cd theme
%   ln -s /usr/share/texlive/texmf-dist/tex/latex/beamer/beamerfontthemeserif.sty beamerfontthemeserif.sty
%%\usepackage{lmodern}
%%\usefonttheme[onlymath]{serif}

\newcommand\crule[3][black]{\textcolor{#1}{\rule{#2}{#3}}}

\title{Super Awesome Project Title}
\author{Max Mustermann\textsuperscript{1}, Alice Eve\textsuperscript{1,2}}
\institute{
    \textsuperscript{1} Helmholtz Center ABC\linebreak
    \textsuperscript{1,2} Helmholtz Center XYZ
}

\begin{document}

\begin{frame}[fragile]
    \begin{columns}[t]
        \begin{column}{.5\linewidth}
            \begin{pblock}{Usage}
                \begin{enumerate}
                    \item Download all files from Github
                    \item Edit \texttt{poster.tex} with your favorite editor
                    \item Compile the slides
                \end{enumerate}
            \end{pblock}

            \begin{pblock}{Typesetting}
                \begin{itemize}
                    \item Standard bullet \textbf{point} can be created with the \texttt{itemize} \textit{environment}
                    \item They can have multiple sub-point
                    \begin{itemize}
                        \item As can be seen here
                        \begin{itemize}
                            \item Or here
                        \end{itemize}
                        \item The ordering is unimportant
                    \end{itemize}
                \end{itemize}
            \end{pblock}

            \begin{pblock}{Equations}
                \begin{equation*}
                    f(x) = \sum_i wx_i^2 + \frac{\beta}{2}
                \end{equation*}
            \end{pblock}

            \begin{pblock}{Figures}
                \centering
                \includegraphics[width=0.3\textwidth]{logos/kit.pdf}
            \end{pblock}
            
            \begin{pblock}{Source Code}
                \begin{minted}[fontsize=\small]{python}
import numpy as np

def foo(a, b):
    """
    asd
    """
    return a + b + 1
                \end{minted}
            \end{pblock}
            
            \begin{pblock}{Citations}
                \fullcite{debus2023reporting}
            \end{pblock}
        \end{column}
        
        \begin{column}{.5\linewidth}

            \begin{pblock}{Colors}
                \begin{table}
                    \small
                    \begin{tabularx}{\textwidth}{cX}
                        \toprule
                        Color & Name\\\midrule
                        \crule[hgfblue]{20pt}{20pt} & hgfblue \\
                        \crule[hgflightblue]{20pt}{20pt} & hgflightblue \\
                        \crule[hgfdarkblue]{20pt}{20pt} & hgfdarkblue \\
                        \crule[hgfmint]{20pt}{20pt} & hgfmint \\
                        \crule[hgfhighlight]{20pt}{20pt} & hgfhighlight \\
                        \crule[hgfpale]{20pt}{20pt} & hgfpale \\
                        \crule[hgfgreen]{20pt}{20pt} & hgfgreen \\
                        \crule[hgfgray]{20pt}{20pt} & hgfgray \\
                        \crule[hgfaerospace]{20pt}{20pt} & hgfaerospace (short: hgfast) \\
                        \crule[hgfearthandenvironment]{20pt}{20pt} & hgfearthandenvironment (short: hgfee) \\
                        \crule[hgfenergy]{20pt}{20pt} & hgfenergy \\
                        \crule[hgfhealth]{20pt}{20pt} & hgfhealth \\
                        \crule[hgfinformation]{20pt}{20pt} & hgfinformation (short: hgfinfo) \\
                        \crule[hgfmatter]{20pt}{20pt} & hgfmatter \\\bottomrule
                    \end{tabularx}
                \end{table}
            \end{pblock}

            \begin{pblock}{Shading}
                \begin{table}
                    \centering
                    \small
                    \begin{tabularx}{\textwidth}{cX}
                        \toprule
                        Color & Name\\\midrule
                        \crule[hgfblue10]{20pt}{20pt} & hgfblue10 \\
                        \crule[hgfblue20]{20pt}{20pt} & hgfblue20 \\
                        \crule[hgfblue30]{20pt}{20pt} & hgfblue30 \\
                        \crule[hgfblue40]{20pt}{20pt} & hgfblue40 \\
                        \crule[hgfblue50]{20pt}{20pt} & hgfblue50 \\
                        \crule[hgfblue60]{20pt}{20pt} & hgfblue60 \\
                        \crule[hgfblue70]{20pt}{20pt} & hgfblue70 \\
                        \crule[hgfblue80]{20pt}{20pt} & hgfblue80 \\
                        \crule[hgfblue90]{20pt}{20pt} & hgfblue90 \\
                        \crule[hgfblue]{20pt}{20pt} & hgfblue \\\bottomrule
                    \end{tabularx}
                \end{table}
            \end{pblock}

            \begin{pblock}{URLs and Fonts}
                \begin{itemize}
                    \item There are raw links with the full URL \url{https://www.google.com}
                    \item You can add also links with names \href{https://www.google.com}{Google}\\~

            		\item You might also want to write in \hermann{Hermann Bold} - Helmholtz's title font
                \end{itemize}
            \end{pblock}
        \end{column}
    \end{columns}

    % optional argument regulates the size of each of the included logos
    \collaborators[1.0em]{{
        logos/dlr.pdf,
        logos/fzj.pdf,
        logos/helmholtz-munich.pdf,
        logos/hereon.pdf,
        logos/hzdr.pdf,
        logos/kit.pdf
    }}
\end{frame}

\end{document}
